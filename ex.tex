% Options for packages loaded elsewhere
\PassOptionsToPackage{unicode}{hyperref}
\PassOptionsToPackage{hyphens}{url}
%
\documentclass[
]{article}
\usepackage{amsmath,amssymb}
\usepackage{lmodern}
\usepackage{iftex}
\ifPDFTeX
  \usepackage[T1]{fontenc}
  \usepackage[utf8]{inputenc}
  \usepackage{textcomp} % provide euro and other symbols
\else % if luatex or xetex
  \usepackage{unicode-math}
  \defaultfontfeatures{Scale=MatchLowercase}
  \defaultfontfeatures[\rmfamily]{Ligatures=TeX,Scale=1}
\fi
% Use upquote if available, for straight quotes in verbatim environments
\IfFileExists{upquote.sty}{\usepackage{upquote}}{}
\IfFileExists{microtype.sty}{% use microtype if available
  \usepackage[]{microtype}
  \UseMicrotypeSet[protrusion]{basicmath} % disable protrusion for tt fonts
}{}
\makeatletter
\@ifundefined{KOMAClassName}{% if non-KOMA class
  \IfFileExists{parskip.sty}{%
    \usepackage{parskip}
  }{% else
    \setlength{\parindent}{0pt}
    \setlength{\parskip}{6pt plus 2pt minus 1pt}}
}{% if KOMA class
  \KOMAoptions{parskip=half}}
\makeatother
\usepackage{xcolor}
\usepackage{longtable,booktabs,array}
\usepackage{calc} % for calculating minipage widths
% Correct order of tables after \paragraph or \subparagraph
\usepackage{etoolbox}
\makeatletter
\patchcmd\longtable{\par}{\if@noskipsec\mbox{}\fi\par}{}{}
\makeatother
% Allow footnotes in longtable head/foot
\IfFileExists{footnotehyper.sty}{\usepackage{footnotehyper}}{\usepackage{footnote}}
\makesavenoteenv{longtable}
\usepackage[draft]{graphicx}
\makeatletter
\def\maxwidth{\ifdim\Gin@nat@width>\linewidth\linewidth\else\Gin@nat@width\fi}
\def\maxheight{\ifdim\Gin@nat@height>\textheight\textheight\else\Gin@nat@height\fi}
\makeatother
% Scale images if necessary, so that they will not overflow the page
% margins by default, and it is still possible to overwrite the defaults
% using explicit options in \includegraphics[width, height, ...]{}
\setkeys{Gin}{width=\maxwidth,height=\maxheight,keepaspectratio}
% Set default figure placement to htbp
\makeatletter
\def\fps@figure{htbp}
\makeatother
\setlength{\emergencystretch}{3em} % prevent overfull lines
\providecommand{\tightlist}{%
  \setlength{\itemsep}{0pt}\setlength{\parskip}{0pt}}
\setcounter{secnumdepth}{-\maxdimen} % remove section numbering
\ifLuaTeX
  \usepackage{selnolig}  % disable illegal ligatures
\fi
\IfFileExists{bookmark.sty}{\usepackage{bookmark}}{\usepackage{hyperref}}
\IfFileExists{xurl.sty}{\usepackage{xurl}}{} % add URL line breaks if available
\urlstyle{same} % disable monospaced font for URLs
\hypersetup{
  hidelinks,
  pdfcreator={LaTeX via pandoc}}

\author{}
\date{}

\begin{document}

\title{\Large\textbf{Análisis de Ciclo de Vida}}
\author{Un Ejemplo Ilustrativo de la Producción de Electricidad}
\date{Presentado por Gustavo Larrea-Gallegos \\ October 27, 2025}
\maketitle

\newpage

\section*{Esquema}

\begin{enumerate}
\item Introducción al Análisis de Ciclo de Vida (ACV)
\item Fase 1: Definición del Objetivo y Alcance
\item Fase 2: Inventario de Ciclo de Vida (ICV)
\item Fase 3: Evaluación de Impacto de Ciclo de Vida (EICV)
\end{enumerate}

\newpage

\section*{¿Qué es el ACV?}

El Análisis de Ciclo de Vida (ACV) es una metodología que evalúa los impactos ambientales pot
enciales de un producto, proceso o servicio durante todo su ciclo de
 vida, desde la extracción de materias primas hasta la disposición final.


Proporciona un marco matemático riguroso para el ACV.

\newpage

\section*{Objetivo y Alcance de Nuestro Ejemplo}

\textbf{Objetivo:} Evaluar el impacto ambiental de la producción de 1 kWh de electricidad.

\textbf{Alcance:} Incluye tres procesos principales:
\begin{itemize}
\item Producción de electricidad
\item Producción de combustible
\item Extracción de petróleo
\end{itemize}

\
textbf{Impactos considerados:} Cambio climático (kg CO₂ eq) y Potencial de Calentamiento Global (PCG).

\newpage

\section*{Procesos Unitarios y Flujos}

Para nuestro análisis, consideraremos tres procesos unitarios interconectados, cada uno con sus respectivos flujos de entrada y salida. Este enfoque nos permite un análisis detallado y ilustrativo.

\newpage

\section*{Desglosando el ICV: Matrices y Vectores}

En la metodología de ACV, el Inventario de Ciclo de Vida se puede representar matemáticamente utilizando matrices y vectores. Esta representación nos permite calcular cuánto necesita operar cada proceso para satisfacer la demanda final.

\newpage

\section*{Matriz Tecnológica (A)}

La matriz tecnológica A describe las relaciones técnicas entre los procesos. Los elementos positivos representan productos principales, mientras que los negativos representan insumos.

\begin{center}
\begin{tabular}{lccc}
\toprule
& Prod. Elec. & Prod. Comb. & Extr. Petr. \\
\midrule
Electricidad & 1000 & -100 & -50 \\
Combustible & -200 & 800 & -60 \\
Petróleo & 0 & -300 & 900 \\
\bottomrule
\end{tabular}
\end{center}

\textbf{Vector de demanda final f:}

\begin{center}
\begin{tabular}{c}
1000 \\
-200 \\
0
\end{tabular}
\end{center}

\newpage

\section*{Matriz de Intervención (B)}

La matriz de intervención B describe los flujos ambientales (emisiones y recursos) asociados con cada proceso unitario.

\begin{center}
\begin{tabular}{lccc}
\toprule
Flujo & Prod. Elec. & Prod. Comb. & Extr. Petr. \\
\midrule
Emisiones CO₂ & 0.5 & 0.3 & 0.2 \\
Uso de agua & 2.0 & 1.0 & 0.5 \\
Uso de tierra & 0.1 & 0.05 & 0.8 \\
\bottomrule
\end{tabular}
\end{center}

(Datos hipotéticos para ilustración)

\newpage

\section*{Cálculo del Vector de Escalamiento}

El vector de escalamiento s se calcula resolviendo el sistema: \textbf{A · s = f}, y luego calcular \textbf{g = B · s}.

Donde:
\begin{itemize}
\item \textbf{A} es la matriz tecnológica
\item \textbf{s} es el vector de escalamiento
\item \textbf{f} es la demanda final
\item \textbf{B} es la matriz de intervención
\item \textbf{g} es el inventario ambiental
\end{itemize}

\newpage

\section*{Resultados del Inventario}

Una vez calculado el vector de escalamiento, podemos determinar el inventario ambiental total para satisfacer nuestra demanda funcional.

\begin{center}
\begin{tabular}{lc}
\toprule
Flujo Ambiental & Cantidad Total \\
\midrule
Emisiones CO₂ (kg) & X \\
Uso de agua (L) & Y \\
Uso de tierra (m²) & Z \\
\bottomrule
\end{tabular}
\end{center}

\newpage

\section*{Matriz de Caracterización}

La matriz de caracterización Q convierte los flujos del inventario en impactos ambientales potenciales.

\begin{center}
\begin{tabular}{lccc}
\toprule
& CO₂ & Agua & Tierra \\
\midrule
Cambio climático & 1.0 & 0 & 0 \\
Eutrofización & 0 & 0.5 & 0.1 \\
Uso de recursos & 0 & 1.0 & 0.8 \\
\bottomrule
\end{tabular}
\end{center}

\newpage

\section*{Evaluación de Impacto}

Los impactos finales se calculan mediante: \textbf{h = Q · g}

Donde:
\begin{itemize}
\item \textbf{Q} es la matriz de factores de caracterización
\item \textbf{g} es el inventario ambiental
\item \textbf{h} es el vector de impactos final
\end{itemize}

\newpage

\section*{Resultados Finales}

Los resultados del ACV nos proporcionan una evaluación cuantitativa de los impactos ambientales asociados con la producción de 1 kWh de electricidad, considerando toda la cadena de suministro.

\begin{center}
\begin{tabular}{lc}
\toprule
Categoría de Impacto & Valor \\
\midrule
Cambio climático (kg CO₂ eq) & A \\
Eutrofización (kg PO₄ eq) & B \\
Uso de recursos (MJ) & C \\
\bottomrule
\end{tabular}
\end{center}

\vfill
\hrule
\vspace{0.5cm}
\textbf{Presentado por Gustavo Larrea-Gallegos \hfill Análisis de Ciclo de Vida \hfill October 27, 2025 \hfill 14 / 14}

\end{document}
